\chapter*{Abstract}
\addcontentsline{toc}{chapter}{Abstract}

We propose an innovative approach for question-answering chatbots to handle conversational contexts and generate natural language sentences as answers. In addition to the ability to answer open-domain questions, our zero-shot learning approach, which uses a pure algorithmic orchestration in a grounded learning manner, provides a modular architecture to swap statically or dynamically task-oriented models while preserving its independence to training.

In the scope of this research, we realize the proof-of-concept of an Open-domain and Closed-ended Question-Answering chatbot able to output comprehensive Natural Language generated sentences using the Wikidata Knowledge Base. 

To achieve the concept, we explore the extraction, and the use of sub-knowledge graphs from the Wikidata knowledge base to answer questions conversationally and to use the sub-graphs as context holder. Additionally, we are extracting Subject-Predicate-Object tuples from the graph and using Language Models to join the SPOs and extend the answers as natural language sentences.

The proof-of-concept architecture uses a combination of state-of-the-art and industry-used models with a fine-tuning strategy. As a motivational target, we use a Zero-Shot Learning approach, by combining various models with an algorithmic orchestrator and using pure algorithmic for the graph manipulation and answer extraction.

Finally, we evaluate the answers and compare the results with state-of-the-art Single-Hop and Multi-Hop question-answering systems on question-answering datasets. We find out that, aside from the computation time and the computational resources needed, our proof-of-concept performs similarly at question-answering compared to its competitors. 

\vskip0.5cm
\textbf{Keywords:} 
\Keywords