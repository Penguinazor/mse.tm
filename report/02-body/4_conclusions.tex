\chapter{Conclusions}
\label{chap:final-conclusions}

\subsection{Project Management}


\section{Final words}
\todo{
Say that it looks like the more the concept or the model is simple, the best it is. It is a nice comparison at how nature works. The simplest survives the best.

Say that transformer are currently leading, their architecture is relatively simple, that's maybe why it's working so well. Say that with addition work, the multiple brains strategy / grounded tasks used by GraphQA can be also seen as the logical simplicity by breaking down difficult tasks into smaller tasks. And the field of study should be explored further in NLP and combined with other fields like Machine Vision, or Sensory Robotics, to build a Multi-Domain Grounded Task Generation model. Setting new standard in machine reasoning and understanding with grounded symbolics.

Additionally, subgraphs are very close to how humans think, we believe that the field of knowledge graph must be explored even further.

Note that our work can later one be adapted to ML by training models to perform at similar tasks as ours. Additionally, we wanted to prove in a first step that the concept is working.

The approach used in this paper is to have analogically to human, a reasoning and the ability to talk. We are using multiple modules to accomplish this is a composite architecture.

}