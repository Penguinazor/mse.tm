\chapter{Introduction}
\label{chap:introduction}

New technologies are revolutionizing the way humans access knowledge as a service from multiple platforms and providers. Thanks to the emergence of increasingly powerful \gls{ai} algorithms, particularly in the field of \gls{nlp}, conversational agents, commonly named chatbots, have come a long way and have become popular among information consumers. As it is in early 2020, chatbots are all still \glspl{ani}\footnote{The State of AI Report 2019 \autocite{studies:state_of_ai_2019}}. Even if the chatbots are continually improving at providing the best outputs for specific tasks as well as providing meaningful human-like sentences, they still cannot generalize the tasks toward human-like conversations. The task of conversation, as humans are applying it, a complex integration of tasks including understanding, reasoning, context linking, context tracking, curiosity, initiatives, \gls{few-shot} or \gls{zero-shot} and learning on the fly, have yet to be accomplished. Nonetheless, as research progresses, chatbots are improving with new technics and tools that are making them step by step closer to complete human-like discussions, slowly progressing towards \gls{agi} chatbots. As for the scope of the thesis, we are humbly focusing on the combination of few \gls{nlp} tasks with a \gls{zero-shot} approach to help \gls{ml} and \gls{nlp} research getting closer to General \gls{qa} Conversational Chatbots. 

\section{Aim of the Research}
The initial goal of the thesis was to explore and combine \gls{sota} \gls{qa} Systems and \glspl{lm} to into an experimental \gls{poc} of a Conversational \gls{qa} Chatbots.

During our research journey, we discovered a new purpose to the project, and took a step into the unknown with a \gls{zero-shot} approach with sub-knowledge graphs.

\subsection{Project's Overall Scope}
We are focusing on the English language as an attempt to increase the number of compatible datasets and make community accessible solutions. We are exploring and combining two types of systems as an attempt to build \gls{qa} chatbots. The first system will produce factual answers, and the second system will generate human-like sentences from the answers found by the primary system. For the factual answers, we will be evaluating the results of our combined system against \gls{sota} \gls{qa} systems on \gls{qa} testing datasets. Humans will manually evaluate the answered sentences from our combined system. Finally, as the time allocated for the thesis is 19 weeks, the outcomes are narrowed at providing non-exhaustive research and a \gls{poc} solution. On a side note, the review of the risks and ethical problems that could be raised by the development of such solutions are not part of this work.

%\newpage

\subsection{Industrial Interest}
\textit{iCoSys}, the Institut of Complex Systems at the University of Applied Sciences and Arts at Fribourg, Switzerland, is interested in the results of this study for their \textit{AI-News} project\footnote{\url{AINews.ch}}. Its goal is to provide a chatbot-based system as a tool for press readers, to help them narrow their interests and deliver the right information. This project is in collaboration with the \textit{Swiss Innovation Agency} from the Swiss Confederation, \textit{La Liberté}, the daily newspaper from Fribourg and \textit{Djebots}, a startup selling scenario-based narrow chatbots.

\subsection{Personal Interest}
In harmony with the thesis subject, as the author is particularly interested in exploring the premises to \gls{agi} related technologies such as \gls{zero-shot}, \gls{gl}, Machine Understanding, and Machine Reasoning for a Multi-Domain Task Generalization. The human-like \gls{qa} frame of this project is particularly motivational.

\section{Research Questions}
We articulate here the initial set of questions as a driver to our research work. From these questions are declined objectives, and from objectives are declined milestones framing the plan.

\begin{itemize}[noitemsep]
    \item What are the components to make \gls{qa} chatbots?
    \begin{itemize}[noitemsep]
        \item What is the \gls{sota} of chatbots and \gls{qa} systems?
        \item How to tune \gls{qa} chatbots to make them as human-like as possible?
        \item How to tune such systems for the field of journalism?
    \end{itemize}
    \item What is the state of the art for \gls{generative} \gls{qa} chatbots?
    \begin{itemize}[noitemsep]
        \item What are the components to make \gls{generative} \gls{qa} chatbots?
        \item Are \gls{generative} chatbots only as good as the data they consume?
        \item Could \gls{generative} chatbots be a step toward \gls{agi}?
    \end{itemize}
\end{itemize}

    %\item Are Knowledge-based Systems useful for \gls{qa} systems?
    %\begin{itemize}[noitemsep]
    %\item Are Knowledge Bases / Knowledge Graphs a  for \gls{qa}?
    %\end{itemize}

